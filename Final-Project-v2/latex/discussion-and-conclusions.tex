\section{Discussion and Conclusions}


A lot was learned from this project. Unfortunately for the group, a lot of this learning was about Python's inefficiencies (Ha Ha..). Looking past this, the group learned a great deal about the basics of simulating flows as no one had ever written a simulation of a fluid. This project gave invaluable insight into one way of going about doing this. While there are undoubtedly many more complex pieces of software that can handle simulations that are of a much higher level, it is useful to understand how the governing equations can be manipulated into equations that can be used without a PDE solver. \\


The simulation of Part 1 took 11 hours to run 60,000 time steps. Part 2 took a little bit more than 7 hours to run 40,000 time steps. This was about 30 times better than if Python was used without the non-trivial Python optimization methods. This is a big reason as to why we would strongly recommend that future students do this project in MATLAB as opposed to Python. The run time was able to be cut down by a factor of 3 by only saving a picture of the simulation every 50 time steps and saving all the data for every time-step that is a multiple of 1,000. A feature was added to the code that allowed for the simulation to start again from any multiple of 1,000 time steps thanks to this method of saving the data. This was useful for moments where our computers let us down in areas such as battery life and logging off due to being idle for too long. \\


The simulation for this project was reasonably valid. In each part, once the flow had reached equilibrium, the heat flux through a surface around the cylinder was tested and compared to the heat flux through a vertical surface near the outflow boundary. Over time, the difference between these values was negligible. This gave the group confidence that the simulation was valid because energy was conserved from the point where it entered the flow to when it exited the flow at the rightmost boundary. This simulation has it's limitations in terms of components that would be introduced in an equivalent 3 dimensional simulation.  \\


One aspect that the group was surprised about was how fast the flow became unsteady and started to have shedding vorticies. The Part 1 simulation only simulates 0.03 seconds of the flow. This is surprising because of how little time it takes for the vorticies to appear and reach a dynamic equilibrium in a flow with a low Reynolds number. The group was also surprised by how relatively cool the flow remained directly behind the cylinder in Part 1. Even at the end of the simulation, the temperature of the fluid at a distance of one third of the radius is still at 300K. This is surprising because of how low the Reynolds number is. From previous work in fluids, the group knew that the flow had not become turbulent and expected it to behave in a way that resembled inviscid flow, since the velocity gradients are low. However, it was seen that this is not the case. Even at low Reynolds numbers the flow will start to create vorticies. \\ 


For the second part of the project, the group chose to run a simulation of a cylinder with a straight, one pixel wide fin on the center line, placed on the downstream side. This line of extra solid material is 25 pixels long (equal to the radius of the circle). This simulation was chosen in order to see the effects it would have on the total heat transfer as well as the St number. This is applicable to lots of situations, which is part of the reason the group found it interesting. If it increases the total heat transfer, it would be a quick way of improving a simple heat exchanger in order to make it more effective. The group found that the average wattage increased from 6.6W in Part 1 to 9.0W in Part 2. This was calculated using the temperature values from the simulation and equation (27). This is a very significant change that has been caused by the fin and is useful in a number of practical applications. Another surprising feature of the flow in Part 2 is that it started to form shedding vorticies at t = 0.0106s, this is 0.0022s later than it took in Part 1. This could explain part of why the average wattage was so different. It seems like the fin that was added has prevented the flow from forming vorticies easily. The group believes this is due to the fact that the solid barrier stops the flow from crossing the center line until further downstream. This is relevant for situations where shedding is an issue. For example, very small heat exchangers, where even small forces can affect the structural integrity of the exchanger. Vortex shedding creates shear forces perpendicular to the inflow direction which can result in failure if not accounted for. The fin has also served the purpose of keeping the flow's velocity high on the downstream side of the cylinder. This can be seen by simply comparing Figure 1 and Figure 2. \\ 

We can see from comparing Figure 2 and Figure 5 that the temperature fluctuates at the same frequency in the two simulations but in Figure 5 the temperature peaks are higher at locations downstream than their counterparts from Figure 2. This reflects the higher level of heat transfer taking place in the $2^{\text{nd}}$ simulation that we ran. By comparing Figures 3 and 6 we can see that they are very similar. This tells us that the fin has little effect on the flow's velocity overall, which makes sense from the perspective of mass conservation. The fin does however speed up the flow on the downstream edge of the cylinder. This 'tactical' flow acceleration makes a big difference in the overall heat transfer value.  \\  

A possible next step for this simulation is to simulate in 3D or to change the simulation to incorporate compressible and high Reynolds number flows. We have made a number of simplifications in order to write and execute this code in a manageable time frame but this takes away from the versatility of the simulation. 







\clearpage